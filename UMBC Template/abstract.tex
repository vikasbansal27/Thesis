\newpage
\pagestyle{empty}

\begin{center}
\vspace{0.1in}
\large{\bf ABSTRACT} \par  
\bigskip \bigskip
\end{center}

\begin{flushleft}
{\bf Title of Thesis:} Recoloring Web Pages For Color Vision Deficiency Users.\\
Vikas Bansal, Masters in Science, 2014 \\
\begin{singlespace}
{\bf Thesis directed by:}{\hspace{2.5mm}} \parbox[t]{3in}{Dr. Lina Zhou, Associate Professor\\
Department of Information Systems}
\end{singlespace}
\begin{singlespace}
{\hspace{37.9mm}}\parbox[t]{3in}{Dr. Tim Finin, Professor\\
Department of Computer Science and \\ Electrical Engineering}
\end{singlespace}
\end{flushleft}

Color vision begins with the activation cone cells. When one of the cone cells dysfunction, color vision deficiency (CVD) ensues. Due to CVD, users become unable to differentiate as many colors a normal person can. Lack of this ability results in less rich web experience, incomprehension of basic information and thus frustration. Solutions such as carefully choosing colors while designing or recolor web pages for CVD users exist. We first present the improvement in the time complexity of an existing tool SPRWeb[6] to recolor web pages. After that we present our tool which explores the foreground-background relationship between colors in a web page. Using this relationship we propose an algorithm which preserves naturalness, pair-differentiability and subjectivity. In the last part, we add an additional step in to algorithm to ensure that the contrast in the parsed  color pairs meets the required W3C guidelines[7]. In evaluation, we found that our algorithm does significantly better in preserving pair-differentiability and produces lower total cost solutions than SPRWeb. Quantitative experimentation of modified algorithm shows that contrast ratio in each replacement pair is more than 4.5 as required for readability.

\par\vfil

