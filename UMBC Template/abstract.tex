\newpage
\pagestyle{empty}

\begin{center}
\vspace{0.1in}
\large{\bf ABSTRACT} \par  
\bigskip \bigskip
\end{center}

\begin{flushleft}
{\bf Title of Thesis:} Recoloring Web Pages For Color Vision Deficiency Users.\\
Vikas Bansal, Masters in Science, 2014 \\
\begin{singlespace}
{\bf Thesis directed by:}{\hspace{2.5mm}} \parbox[t]{3in}{Dr. Lina Zhou, Associate Professor\\
Department of Information Systems}
\end{singlespace}
\begin{singlespace}
{\hspace{37.9mm}}\parbox[t]{3in}{Dr. Tim Finin, Professor\\
Department of Computer Science and \\ Electrical Engineering}
\end{singlespace}
\end{flushleft}

Colors are an important part of our life. They are commonly used to represent important information, specially, categories. Ability to differentiate between colors is important in performing routine tasks such as reading content, following traffic lights etc. This ability varies from person to person. Many people experience difficulty in reading content on web pages due to this variation. These difficulties result from the inability of individuals to sufficiently differentiate between colors. This condition of an individual is called Color Vision Deficiency (CVD). More than four percent of current population suffer from some kind of CVD, significantly affecting their web experience.

To improve the web experience of CVD users, we have presented an algorithm which can be used to recolor web pages such that the recolored web pages do not pose any difficulty to a CVD user. Replaced colors are chosen from a fixed set of color called Dichromacy Trichromacy Equivalency Plane (DTEP) set. While recoloring we also preserve the naturalness and contrast among foreground and background colors in different sections of the web page. A quantitative comparison with the existing tool SPRWeb[] shows that our algorithm performs better in preserving contrast among different sections of the web pages and doesn’t differ much in preserving naturalness.

An additional step in to algorithm was added to induce the contrast in pairs according to the W3C guidelines. Quantitative experimentation of modified algorithm shows that contrast ratio in each replacement pair is more than 4.5 as required for readability. 

\par\vfil

